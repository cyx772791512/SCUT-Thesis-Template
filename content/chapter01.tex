\chapter{绪论}
%
\section{研究背景和意义}
\subsection{研究背景和意义}
%
关于\LaTeX{}以及基于\LaTeX{}写作的好处不再赘述。\LaTeX{}的入门资料推荐文献\parencite{_b}以及文献\parencite{_c}。

这里主要是想推荐一种“学术生态”,即利用各种工具展开科研工作,以达到事半功倍的效果。需要用到以下软件:
\begin{enumerate}
	\item 	参考文献管理软件zotero\cite{_f}。很多人使用过endnote,但其实zotero也非常强大,强烈推荐。可到b站观看Struggle with Me出品的视频教程\cite{_e}入门。\replaced{最新的zotero已支持软件内部PDF阅读。}{zotero不自带pdf阅读器,使用Adobe Acrobat pro DC即可。
				在Adobe中点击文件->属性->位置,即可打开文件所在位置,故亦不推荐更改zotero的文件系统。}
	\item	可截图获取文献中公式的软件mathpix\cite{_d}。在阅读别人的论文时,很可能需要把文章中的公式抄下来放到自己的笔记中,方便以后组会报告甚至论文中使用,这时使用mathpix可直接截图获取\LaTeX{}源码,非常方便。该软件普通邮箱注册可每月50次免费,学校邮箱可100次,若信用卡注册可1000次。	
	\item	\replaced{VS Code}{TeXstudio},相当于IDE。本模板是基于\replaced{VS Code}{TeXstudio2020}进行的,关于该软件的使用(快捷键等)可另行查找资料。\deleted{编译时可以使用该软件,也可以运行文件目录的all.bat。TeXstudio的设置见第二章。}
\end{enumerate}

\added{本文总体结构以及前三章均继承了前辈们的框架与内容,本人所做修改均标注为蓝色,第四章为本人撰写,故未作颜色标记。}本文的章节安排如下:

第一章,绪论。

第二章,模板简介。主要介绍各文件的内容。

第三章,常用环境。介绍论文写作中常用的环境,包括:图、表、公式、定理。基本涵盖了常用的命令。

\added{第四章,本人对前辈们的模版的修改之处,包括:双盲评审、图、公式、参考文献、标题等。}



