\chapter{攻读博士学位期间取得的研究成果}

一、已发表(包括已接受待发表)的论文,以及已投稿、或已成文打算投稿、或拟成文投稿的论文情况\underline{\textbf{(只填写与学位论文内容相关的部分):}}

\begin{centering}
	\small
	\begin{longtable}{|>{\centering}m{0.5cm}|>{\centering}m{2.2cm}|>{\centering}m{3.3cm}|>{\centering}m{2.7cm}|>{\centering}m{1.8cm}|>{\centering}m{1.8cm}|>{\centering}m{1cm}|}
		\hline 
		\textbf{序号} & \textbf{作者} & \textbf{题\qquad 目} 						   & \textbf{发表或投稿刊物名称、级别} & \textbf{发表的卷期、年月、页码} & \textbf{与学位论文哪一部分(章、节)相关} & \textbf{被索引收录情况}\tabularnewline
		\hline 
		1   & \textbf{Cui Yanxin}, Kang Youping,\\ Shi Yonghua, Chen Jinrong, Wang Zishun, Wang Jinyi & Investigation into the arc profiles and penetration ability of axial magnetic field-enhanced K-TIG welding by means of a specially designed sandwich & Journal of Manufacturing Processes\\(IF:5.01, JCR Q2) & 2021, 68:32-41 & 第五章 & SCI\tabularnewline
		\hline 
		2	& \textbf{Cui Yanxin}, Shi Yonghua, Hong Xiaobin & To be continued & Journal of Manufacturing Processes\\(IF:5.01, JCR Q2)  & 2019, 46:225-233 & 第六章 &SCI \tabularnewline
		\hline
	\end{longtable}
\end{centering}
\newpage
二、与学位内容相关的其它成果(包括专利、著作、获奖项目等)

1、与学位内容相关的著作

\begin{centering}
	\small
	\begin{longtable}{|>{\centering}m{0.5cm}|>{\centering}m{2.2cm}|>{\centering}m{4.7cm}|>{\centering}m{2.7cm}|>{\centering}m{1.8cm}|>{\centering}m{1.8cm}|}
		\hline 
		\textbf{序号} & \textbf{作者} & \textbf{著作名称} 						   & \textbf{出版社} & \textbf{出版的年月、页码} & \textbf{与学位论文哪一部分(章、节)相关} \tabularnewline
		\hline 
		1	& Shi Yonghua, \textbf{Cui Yanxin}, Cui Shuwan, Zhang Baori & A Novel High-Efficiency Keyhole Tungsten Inert Gas (K-TIG) Welding: Principles and Practices & Welding Technology. Cham: Springer International Publishing & 2021: 313–367 & 第三章\\第六章\tabularnewline
		\hline
	\end{longtable}
\end{centering}

2、与学位内容相关的专利

\begin{centering}
	\small
	\begin{longtable}{|>{\centering}m{0.5cm}|>{\centering}m{4cm}|>{\centering}m{3.5cm}|>{\centering}m{4cm}|>{\centering}m{2.2cm}|}
		\hline
		\textbf{序号}	&	\textbf{专利申请人}	&	\textbf{专利名称}	&	\textbf{专利号}	&	\textbf{与学位论文哪一部分(章、节)相关}\tabularnewline
		\hline
		1	& 石永华,\textbf{崔延鑫},陈金荣,陈云可	&	一种用于锁孔效应深熔TIG焊的电弧压力测量装置与方法	&	发明专利202110489846.6(已授权)	& 第二章\tabularnewline
		\hline
		2	& ***	& 	***	& ***	& 第二章\tabularnewline
		\hline
	\end{longtable}
\end{centering}